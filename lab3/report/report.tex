% !TeX spellcheck = ru_RU
\include{settings}
\usepackage{minted}

\begin{document}	% начало документа

% Титульная страница
\include{titlepage}

\newpage
\setcounter{page}{2}
% Содержание
\include{ToC}


\section{Цель работы}
Сформировать набор данных, позволяющий производить операции на реальных объемах данных.

\section{Программа работы}
	\begin {enumerate}
	\item Реализация в виде программы параметризуемого генератора, который позволит сформировать набор связанных данных в каждой таблице.
	\item Частные требования к генератору, набору данных и результирующему набору данных:
	\begin{itemize}
		\item количество записей в справочных таблицах должно соответствовать ограничениям предметной области
		\item количество записей в таблицах, хранящих информацию об объектах или субъектах должно быть параметром генерации
		\item значения для внешних ключей необходимо брать из связанных таблиц
	\end{itemize}
	\end {enumerate}

\section{Выполнение работы}
В качестве языка программирования для параметризуемого создания генератора был выбран Python 3.6 и библиотека psycopg2 - самая популярная библиотека для работы с PostgreSQL.

В ходе выполнения работы была написана программа, реализующая генератор.
Код программы приведён ниже.

\large {Листинг 1. fill.py}
\inputminted[
frame=lines,
framesep=2mm,
baselinestretch=1.2,
fontsize=\footnotesize,
linenos
]{python}{../src/fill.py}

\section {Результат работы}

Запустим данную программу через терминал.

\begin{lstlisting}
(base) grafa@KRAB:~/Desktop/wine_card/lab3/src$ python fill.py
Successfully filled components and drinks!
Successfully filled places!
Successfully filled food!
Successfully filled places_food!
Successfully filled places_drinks!
Successfully filled discounts!
Successfully filled supplies_drinks!
Successfully filled supplies_food!
\end{lstlisting}

Примеры генерируемых данных:

\begin{figure}[H]
	\begin{center}
		\includegraphics[scale=0.3]{./pics/components.png}
		\caption{Содержимое таблицы components} 
		\label{pic:components} % название для ссылок внутри кода
	\end{center}
\end{figure}

\begin{figure}[H]
	\begin{center}
		\includegraphics[scale=0.5]{./pics/supplies_drinks.png}
		\caption{Содержимое таблицы supplies\_drinks} 
		\label{pic:supplies_drinks} % название для ссылок внутри кода
	\end{center}
\end{figure}

\begin{figure}[H]
	\begin{center}
		\includegraphics[scale=0.5]{./pics/discounts.png}
		\caption{Содержимое таблицы discounts} 
		\label{pic:discounts} % название для ссылок внутри кода
	\end{center}
\end{figure}


\newpage
\section{Выводы}

В ходе выполнения данной работы на языке программирования Python был написан параметризуемый генератор.
В качестве параметра данного генератора можно указать количество записей в таблицах, как это и требовало задание.
\end{document}

