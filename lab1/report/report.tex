% !TeX spellcheck = ru_RU
\include{settings}
\usepackage{minted}

\begin{document}	% начало документа

% Титульная страница
\include{titlepage}

% Содержание
\include{ToC}


\section{Цель работы}
Познакомиться с основами проектирования схемы БД, способами организации данных в SQL-БД.

\section{Программа работы}
	\begin {enumerate}
	\item Создание проекта для работы в GitLab.
	\item Выбор задания (предметной области), описание набора данных и требований к хранимым данным в свободном формате в wiki своего проекта в GitLab.
	\item Формирование в свободном формате (предпочтительно в виде графической схемы) cхемы БД, соответствующей заданию. Должно получиться не менее 7 таблиц.
	\item Согласование с преподавателем схемы БД. Обоснование принятых решений и соответствия требованиям выбранного задания. 
	\item Выкладывание схемы БД в свой проект в GitLab.
	\item Демонстрация результатов преподавателю.
	\end {enumerate}

\section{Выполнение работы}


\subsection{Выбор предметной области}

В качестве предметной области был выбран каталог баров. База данных хранит перечень баров, а также еды и напитков.

\subsection{Описание таблиц}
\begin{itemize}
	\item components - ингредиенты для напитков. Имеет поля, в которых указывается название и процент алкоголя.
	\item drinks - напитки. Поля: название, рейтинг, объём, процент алкоголя, средняя цена, тип напитка.
	\item places - заведения. Поля: название, адрес, рейтинг, средний счёт.
	\item food - еда. Поля: название, рейтинг, объём, средняя цена.
	\item discounts - скидки. Поля: заведение(внешний ключ), тип напитка, процент скидки, описание, день недели, время начала, время конца.
	\item supplies\_drinks - поставки напитков. Поля: заведение(внешний ключ), идентефикатор напитка, количество, цена за еденицу, дата.
	\item supplies\_food -поставки еды. Поля: заведение(внешний ключ), идентефикатор еды, количество, цена за еденицу, дата.
	\item Вспомогательные: 
		\begin {itemize}
		\item components\_drinks
		\item places\_drinks
		\item places\_food
	\end {itemize}
\end{itemize}


\subsection{Структура базы данных}

\begin{figure}[H]
	\begin{center}
		\includegraphics[scale=0.3]{../structure.png}
		\caption{Структура базы данных} 
		\label{pic:struct} % название для ссылок внутри кода
	\end{center}
\end{figure}

\section{Выводы}

В ходе выполнения данной работы была спроектирована и согласована с преподавателем база данных, содержащая перечень баров, а также еды и напитков.

\end{document}

