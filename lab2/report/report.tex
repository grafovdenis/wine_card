% !TeX spellcheck = ru_RU
\include{settings}
\usepackage{minted}

\begin{document}	% начало документа

% Титульная страница
\include{titlepage}

\newpage
\setcounter{page}{2}
% Содержание
\include{ToC}


\section{Цель работы}
Познакомиться с основами проектирования схемы БД, языком описания сущностей и ограничений БД SQL-DDL.

\section{Программа работы}
	\begin {enumerate}
	\item Самостоятельное изучение SQL-DDL.
	\item Создание скрипта БД в соответствии с согласованной схемой. Должны присутствовать первичные и внешние ключи, ограничения на диапазоны значений. Демонстрация скрипта преподавателю. 
	\item Создание скрипта, заполняющего все таблицы БД данными.
	\item Выполнение SQL-запросов, изменяющих схему созданной БД по заданию преподавателя. Демонстрация их работы преподавателю.
	\end {enumerate}

\section{Теоретическая информация}

\subsection{Язык SQL}
Язык SQL (Structured Query Language) - язык структурированных запросов. Он позволяет формировать весьма сложные запросы к базам данных. В SQL определены два подмножества языка:
\begin {itemize}
	\item SQL-DDL (Data Definition Language) - язык определения структур и ограничений целостности баз данных. Сюда относятся команды создания и удаления баз данных; создания, изменения и удаления таблиц; управления пользователями и т.д.
	\item SQL-DML (Data Manipulation Language) - язык манипулирования данными: добавление, изменение, удаление и извлечение данных, управления транзакциями
\end {itemize}

\subsection{Операции ON DELETE и ON UPDATE}
Выражения ON DELETE и ON UPDATE внешних ключей используются для указания действий, которые будут выполняться при удалении строк родительской таблицы (ON DELETE) или изменении родительского ключа (ON UPDATE). Один и тот же внешний ключ может иметь различные действия, указанные для ON DELETE и ON UPDATE.

В базе данных PostgreSQL действия ON DELETE и ON UPDATE, ассоциированные с внешним ключом, могут быть следующими: NO ACTION, RESTRICT, SET NULL, SET DEFAULT или CASCADE. Если действие не указывается специально, оно по умолчанию является NO ACTION.
Рассмотрим действия, используемые в данной лабораторной работе.

\begin{itemize}
	\item \textbf{NO ACTION}: опция «NO ACTION» означает, что когда родительский ключ изменяется или удаляется из базы данных, никаких специальных действий не производится.
	\item \textbf{CASCADE}: действие «CASCADE» распространяет операции удаления и изменения родительского ключа на зависящие от него дочерние ключи. Для действия ON DELETE CASCADE это выражается в том, что каждая строка дочерней таблицы, которая ассоциирована с удаляемой родительской строкой, также будет удалена. Для действия ON UPDATE CASCADE это выражается в том, что значения, сохранённые в зависящем дочернем ключе, будут заменены на новые значения родительского ключа.
\end{itemize}

\subsection{Типы данных}
Символьные типы данных:
\begin {itemize}
	\item CHAR(n) - символьные строки фиксированной длины. Длина строки определяется параметром n. CHAR без параметра соответствует CHAR(1). Для хранения таких данных всегда отводится n байт вне зависимости от реальной длины строки.
	\item VARCHAR(n) - символьная строка переменной длины. Для хранения данных этого типа отводится число байт, соответствующее реальной длине строки.
\end {itemize}

Целые типы данных:
\begin {itemize}
	\item SMALLINT - короткое целое (2 байта)
	\item INTEGER - обычное целое (4 байта) 
	\item BIGINT - длинное целое (8 байт) 
\end {itemize}

Вещественные типы данных:
\begin {itemize}
	\item REAL - числа с плавающей точкой (4 байта).
	\item DOUBLE PRESCISION - числа с плавающей точкой (8 байт).
	\item NUMERIC(p,n) - тип данных аналогичный FLOAT с числом значащих цифр p и точностью n.
\end {itemize}

Дата и время - используются для хранения даты, времени и их комбинаций:
\begin {itemize}
	\item DATE - тип данных для хранения даты.
	\item TIME - тип данных для хранения времени.
	\item TIMESTAMP - тип данных для хранения моментов времени (год + месяц + день + часы + минуты + секунды + доли секунд).
\end {itemize}

Двоичные типы данных - позволяют хранить данные любого объема в двоичном коде (оцифрованные изображения, исполняемые файлы и т.д.):
\begin {itemize}
	\item BYTEA
\end {itemize}

\section{Выполнение работы}

\subsection {Инициализация}

\begin{center}
	Листинг 1: init.sql
\end{center}
\inputminted[
frame=lines,
framesep=2mm,
baselinestretch=1.2,
fontsize=\footnotesize,
linenos
]{sql}{../src/init.sql}

\subsection {Заполнение}

\begin{center}
	Листинг 2: fill.sql
\end{center}
\inputminted[
frame=lines,
framesep=2mm,
baselinestretch=1.2,
fontsize=\footnotesize,
linenos
]{sql}{../src/fill.sql}

\subsection {Изменение}
Формулировка требуемых изменений:
\begin{enumerate}
	\item Добавить цены на позиции меню в каждом конкретном баре.
	\item Добавить поставки блюд и ингредиентов для коктейлей в бары.
\end{enumerate}

\begin{center}
	Листинг 3: change.sql
\end{center}
\inputminted[
frame=lines,
framesep=2mm,
baselinestretch=1.2,
fontsize=\footnotesize,
linenos
]{sql}{../src/change.sql}

\section {Результат работы}

\begin{figure}[H]
	\begin{center}
		\includegraphics[scale=0.3]{../../structure.png}
		\caption{Структура базы данных} 
		\label{pic:struct} % название для ссылок внутри кода
	\end{center}
\end{figure}

\begin{figure}[H]
	\begin{center}
		\includegraphics[scale=0.5]{./pics/drinks.png}
		\caption{Содержимое таблицы drinks} 
		\label{pic:drinks} % название для ссылок внутри кода
	\end{center}
\end{figure}

\begin{figure}[H]
	\begin{center}
		\includegraphics[scale=0.5]{./pics/places.png}
		\caption{Содержимое таблицы places} 
		\label{pic:places} % название для ссылок внутри кода
	\end{center}
\end{figure}

\begin{figure}[H]
	\begin{center}
		\includegraphics[scale=0.5]{./pics/places_drinks.png}
		\caption{Содержимое таблицы places\_drinks} 
		\label{pic:places_drinks} % название для ссылок внутри кода
	\end{center}
\end{figure}

\newpage
\section{Выводы}

В ходе выполнения данной работы были написаны 3 скрипта на языке PostgreSQL: создающий таблицы; наполняющий таблицы данными; изменяющий таблицы и добавляющий новые.
Таким образом, было осуществлено моё знакомство с основами проектирования схемы БД, языком описания БД SQL-DDL.

\end{document}

