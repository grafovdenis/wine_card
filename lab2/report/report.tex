% !TeX spellcheck = ru_RU
\include{settings}
\usepackage{minted}

\begin{document}	% начало документа

% Титульная страница
\include{titlepage}

% Содержание
\include{ToC}


\section{Цель работы}
Познакомиться с основами проектирования схемы БД, языком описания сущностей и ограничений БД SQL-DDL.

\section{Программа работы}
	\begin {enumerate}
	\item Самостоятельное изучение SQL-DDL.
	\item Создание скрипта БД в соответствии с согласованной схемой. Должны присутствовать первичные и внешние ключи, ограничения на диапазоны значений. Демонстрация скрипта преподавателю. 
	\item Создание скрипта, заполняющего все таблицы БД данными.
	\item Выполнение SQL-запросов, изменяющих схему созданной БД по заданию преподавателя. Демонстрация их работы преподавателю.
	\end {enumerate}

\section{Теоретическая информация}

\subsection{Язык SQL}
Язык SQL (Structured Query Language) - язык структурированных запросов. Он позволяет формировать весьма сложные запросы к базам данных. В SQL определены два подмножества языка:
\begin {itemize}
	\item SQL-DDL (Data Definition Language) - язык определения структур и ограничений целостности баз данных. Сюда относятся команды создания и удаления баз данных; создания, изменения и удаления таблиц; управления пользователями и т.д.
	\item SQL-DML (Data Manipulation Language) - язык манипулирования данными: добавление, изменение, удаление и извлечение данных, управления транзакциями
\end {itemize}

\subsection{Типы данных}
Символьные типы данных:
\begin {itemize}
	\item CHAR(n) - символьные строки фиксированной длины. Длина строки определяется параметром n. CHAR без параметра соответствует CHAR(1). Для хранения таких данных всегда отводится n байт вне зависимости от реальной длины строки.
	\item VARCHAR(n) - символьная строка переменной длины. Для хранения данных этого типа отводится число байт, соответствующее реальной длине строки.
\end {itemize}

Целые типы данных:
\begin {itemize}
	\item SMALLINT - короткое целое (2 байта)
	\item INTEGER - обычное целое (4 байта) 
	\item BIGINT - длинное целое (8 байт) 
\end {itemize}

Вещественные типы данных:
\begin {itemize}
	\item REAL - числа с плавающей точкой (4 байта).
	\item DOUBLE PRESCISION - числа с плавающей точкой (8 байт).
	\item NUMERIC(p,n) - тип данных аналогичный FLOAT с числом значащих цифр p и точностью n.
\end {itemize}

Дата и время - используются для хранения даты, времени и их комбинаций:
\begin {itemize}
	\item DATE - тип данных для хранения даты.
	\item TIME - тип данных для хранения времени.
	\item TIMESTAMP - тип данных для хранения моментов времени (год + месяц + день + часы + минуты + секунды + доли секунд).
\end {itemize}

Двоичные типы данных - позволяют хранить данные любого объема в двоичном коде (оцифрованные изображения, исполняемые файлы и т.д.):
\begin {itemize}
	\item BYTEA
\end {itemize}

\section{Выполнение работы}

\subsection {Инициализация}

\inputminted[
frame=lines,
framesep=2mm,
baselinestretch=1.2,
fontsize=\footnotesize,
linenos
]{sql}{../src/init.sql}

\subsection {Изменение}

\inputminted[
frame=lines,
framesep=2mm,
baselinestretch=1.2,
fontsize=\footnotesize,
linenos
]{sql}{../src/changing.sql}

\subsection {Заполнение}

\inputminted[
frame=lines,
framesep=2mm,
baselinestretch=1.2,
fontsize=\footnotesize,
linenos
]{sql}{../src/fill.sql}

\section{Выводы}

В ходе выполнения данной работы были написаны 3 скрипта на языке SQL: создающий таблицы; изменяющий таблицы и добавляющий новые; скрипт, наполняющий таблицы данными.
Таким образом, было осуществлено моё знакомство с основами проектирования схемы БД SQL.

\end{document}

