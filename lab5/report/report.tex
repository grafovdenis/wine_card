\include{settings}

\begin{document}	% начало документа

% Титульная страница
\include{titlepage}

% Содержание
\include{ToC}


\section{Цель работы}
Познакомить студентов с возможностями реализации более сложной обработки данных на стороне сервера с помощью хранимых процедур.

\section{Программа работы}
\begin{itemize}
	\item Изучение возможностей языка PL/pgSQL.
	\item Создание двух хранимых процедур в соответствии с индивидуальным заданием, полученным у преподавателя.
	\item Выкладывание скрипта с созданными сущностями в репозиторий.
	\item Демонстрация результатов преподавателю.
\end{itemize}

\section{Теоретическая информация}
\subsection{PL/pgSQL}
PL/pgSQL (англ. Procedural Language/PostGres Structured Query Language) — процедурное расширение языка SQL, используемое в СУБД PostgreSQL. Этот язык предназначен для написания функций, триггеров и правил и обладает следующими особенностями:

\begin{itemize}
	\item добавляет управляющие конструкции к стандарту SQL;
	\item допускает сложные вычисления;
	\item может использовать все объекты БД, определенные пользователем;
	\item прост в использовании.
\end{itemize}

\subsection{Хранимая процедура}
Хранимая процедура — объект базы данных, представляющий собой набор SQL-инструкций, который компилируется один раз и хранится на сервере. Хранимые процедуры очень похожи на обыкновенные процедуры языков высокого уровня, у них могут быть входные и выходные параметры и локальные переменные, в них могут производиться числовые вычисления и операции над символьными данными, результаты которых могут присваиваться переменным и параметрам. В хранимых процедурах могут выполняться стандартные операции с базами данных (как DDL, так и DML). Кроме того, в хранимых процедурах возможны циклы и ветвления, то есть в них могут использоваться инструкции управления процессом исполнения.

\newpage
\section{Ход работы}
\subsection{Первое индивидуальное задание}
\textit{Для заданного бара и даты процедура выводит список доступных позиций на тот момент в баре. Считается, что товар, поставленный более, чем 2 недели назад, уже был распродан.}

Мной была написана функция get\_available(integer, timestamp), принимающая 2 параметра:
\begin{itemize}
	\item place\_id integer - для указания заведения (PK таблицы places);
	\item from\_date timestamp - для указания даты.
\end{itemize}

Функция возвращает таблицу, содержащую информацию о доступных позициях, а также даты поставки данных позиций.
Запрос, формирующий указанную таблицу, можно разделить на 2 части. Первая часть возвращает все поставки напитков из заданного заведения и за указанный промежуток времени. Вторая часть возвращает все поставки еды с теми же критериями.
Затем, с помощью оператора union, полученные подвыборки объединяются и возвращаются в качестве результата выполнения функции.

\lstinputlisting[
language=SQL,
caption={sp1.sql},
]{../../src/sp1.sql}

\begin{figure}[H]
	\begin{center}
		\includegraphics[scale=0.45]{sp1}
		\caption{Результат работы} 
		\label{pic:1} % название для ссылок внутри кода
	\end{center}
\end{figure}

\subsection{Второе индивидуальное задание}
\textit{Для заданного временного интервала процедура выводит перечень акций в каждом из баров в формате json:}
\begin{lstlisting}
{
	"bar title": [
		{
			"weekday": day,
			"discount": description of discount,
		},
		...
	],
	...
}
\end{lstlisting}

Пояснение: на данный момент структура БД не позволяет указать временной интервал проведения акции, поэтому данная функция не имеет параметров.

Написанная функция get\_discounts() создаёт таблицу disc, содержащую информащию о всех скидках во всех барах.
Затем, с помощью операторов json\_object\_agg и json\_build\_object, из данных таблицы disc формируется результат функции в формате json.

\lstinputlisting[
language=SQL,
caption={sp2.sql},
]{../../src/sp2.sql}

\begin{figure}[H]
	\begin{center}
		\includegraphics[scale=0.5]{sp2}
		\caption{Результат работы} 
		\label{pic:2} % название для ссылок внутри кода
	\end{center}
\end{figure}

\section{Выводы}
В ходе данной работы я ознакомился с процедурным языком PL/pgSQL, изучил его особенности и возможности, которые он предоставляет. Во время выполнения индивидуальных заданий я получил практические навыки по реализации обработки данных на стороне сервера с помощью хранимых процедур. Кроме того, я научился выводить результаты запросов в формате json.
\end{document}
